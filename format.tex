%设置基本class,A4纸
\documentclass[UTF8,a4paper]{ctexart}

%设置基本字体大小,四号14.4磅(pt)
\usepackage[fontsize=14.4]{fontsize}
%\setlength{\baselineskip}{28pt}
%\linespread{1.6}

%设置页面布局,页边距,页眉页脚
\usepackage[top=30mm, bottom=20mm, left=28mm, right=26mm,headsep=0mm,headheight=15mm, footskip=17.5mm]{geometry}

%两端对齐,正文中\justifying
\usepackage{ragged2e}

% 设置页眉、页脚
\usepackage{fancyhdr}
\lhead{}% 页眉左边设为空
\chead{}% 页眉中间设为空
\rhead{}% 页眉右边设为空
\lfoot{}% 页脚左边设为空
\cfoot{\thepage}% 页脚中间显示页码
\rfoot{}% 页脚右边设为空
\renewcommand{\headrulewidth}{0pt}%页眉修饰线宽度0(不显示)
\setlength{\footskip}{0mm}

%设置字体
\usepackage{fontspec}
\setmainfont{Times New Roman}
\setCJKmainfont{宋体}[AutoFakeBold={2}]
%\newCJKfontfamily\songti{宋体}[AutoFakeBold] % 重定义

%各级标题字体,使用伪粗体
\newCJKfontfamily\sectioncf[AutoFakeBold={2}]{黑体}
\newCJKfontfamily\subsectioncf[AutoFakeBold={2}]{楷体}
\newCJKfontfamily\subsubsectioncf[AutoFakeBold={2}]{仿宋}
\newCJKfontfamily\subsubsubsectioncf[AutoFakeBold={2}]{仿宋}
\usepackage{titlesec} %自定义多级标题格式的宏包
%\titleformat{章节命令}[形状]{格式}{标题序号}{序号与标题间距}{标题前命令}[标题后命令]
%一级标题
\titleformat{\section}
    [block]
    {\hspace{2em}\zihao{3}\sectioncf\bfseries}
    {\zihao{3}\chinese{section}、}
    {0pt}
    {}[]
\titlespacing{\section}{0pt}{0pt}{0pt}%段前段后间距0
%二级标题
\titleformat{\subsection}
    [block]
    {\hspace{2em}\zihao{3}\subsectioncf\bfseries}
    {(\chinese{subsection})}
    {0pt}
    {}[]
\titlespacing{\subsection}{0pt}{0pt}{0pt}%段前段后间距0
%三级标题
\titleformat{\subsubsection}
    [block]
    {\hspace{2em}\zihao{4}\subsubsectioncf\bfseries}
    {\arabic{subsection}.\arabic{subsubsection}}
    {0pt}
    {}[]
\titlespacing{\subsubsection}{0pt}{0pt}{0pt}%段前段后间距0

%正文段落
\titleformat{\paragraph}
    [block]
    {\hspace{2em}\zihao{4}}
    {[\arabic{paragraph}]}
    {1em}
    {}[]

%设置目录显示等级到3级标题
\setcounter{tocdepth}{3}
%设置目录样式
\usepackage{titletoc}
\titlecontents{section}[1em]{}{}{}{}[]

\CTEXsetup[format={\zihao{4}\sectioncf}]{section}
%\CTEXsetup[name={第,章}]{section}                            %设置章标题字号为Large,居左
\CTEXsetup[number={ \chinese{section}}]{section}                      %section形式改为一,二,三,..
\CTEXsetup[format={\zihao{-4}}]{section}
\CTEXsetup[name={(,)}]{subsection}
\CTEXsetup[number={\chinese{subsection}}]{subsection}                %subsection形式改为(一,二,三,...)
\CTEXsetup[number=\arabic{subsubsection}]{subsubsection}             %subsubsection形式改为1,2,3,...

\dottedcontents{section}[2em]{\normalsize}{2.0em}{4pt}
\dottedcontents{subsection}[3.0em]{\normalsize}{2.0em}{4pt}
\dottedcontents{subsubsection}[4.0em]{\normalsize}{2.0em}{4pt}


% 设置标题
\usepackage{caption}
\captionsetup[table]{font={bf,footnotesize}, labelsep=space, name=表}% 设置表格编号头
\renewcommand {\thetable} {\thesection{}-\arabic{table}\,}%重定义表序号样式
%\renewcommand {\thefigure} {\thechapter{}.\arabic{figure}}%重定义图序号样式

\usepackage{makecell}
\usepackage{multirow}    %纵向合并单元格
\usepackage{booktabs}    %绘制三线表(\toprule , \midrule , \bottonrule , \cmidrule)
% 横向合并 \multicolumn{⟨n⟩}{⟨column-spec⟩}{⟨item⟩}。其中 ⟨n⟩ 为要合并的列数,⟨column-spec⟩ 为合并单元格后的列格式,只允许出现一个 l/c/r 或p 格式。如果合并前的单元格前后带表格线 |,合并后的列格式也要带 | 以使得表格的竖线一致
% 纵向合并 \multirow{⟨n⟩}{⟨width⟩}{⟨item⟩},⟨width⟩ 为合并后单元格的宽度,可以填*以使用自然宽度

%表格行号自动序号
\newcounter{rowno}
\setcounter{rowno}{0}
